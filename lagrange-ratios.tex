% Default course lecture note template by asp 
\documentclass[letterpaper]{article}
\usepackage[utf8]{inputenc}
\usepackage[T1]{fontenc}
\usepackage[english]{babel}
\usepackage[top=3cm, bottom=3cm, left=3.85cm, right=3.85cm]{geometry}
\usepackage[onehalfspacing]{setspace}
\usepackage{amsmath} 
\usepackage{amssymb}
\usepackage{wasysym}
\usepackage{amsthm}
\usepackage{graphicx}
\usepackage[usenames,dvipsnames]{color}

\begin{document}

% Two-column title block
\begin{minipage}[b]{0.7\linewidth}
{\huge Lagrange Ratio Thing}
\end{minipage}
\begin{minipage}[b]{0.3\linewidth}
  \begin{flushright}
    Andrew Poelstra\\
    2021 November 28
  \end{flushright}
\end{minipage}
\\

% Actual content start
\paragraph{Theorem.} Let $x,y,z$ be distinct elements of a field $\mathbb{F}$,
none equal to a distinguished element $S\in\mathbb{F}$. Define the two quantities
\[
    \alpha = \frac{S - y}{S - x} \qquad
    \beta = \frac{S - z}{S - x}
\]
Then the Lagrange basis polynomials evaluated at $S$
\[ \ell_x(S) = \frac{(S - y)(S - z)}{(x - y)(x - z)} \]
\[ \ell_y(S) = \frac{(S - x)(S - z)}{(y - x)(y - z)} \]
\[ \ell_z(S) = \frac{(S - x)(S - y)}{(z - x)(z - y)} \]
can be computed explicitly in terms of $\alpha$ and $\beta$.

\paragraph{Proof.} We will provide the proof for $\ell_x(S)$. The others
proceed almost identically.

We first observe that
\begin{align*}
S - y &= \alpha(S - x) \\
S - z &= \beta(S - x) \\
x - y &= (S - y) - (S - x) \\
      &= (\alpha - 1)(S - x) \\
x - z &= (S - z) - (S - x) \\
      &= (\beta - 1)(S - x)
\end{align*}
In other words, all four parentetical terms in the expression for $\ell_x(S)$
are equal to $(S - x)$ times some function of $\alpha$ and $\beta$. Putting
them all together,
\[ \ell_x(S) = \frac{\alpha\beta}{(\alpha - 1)(\beta - 1)} \]

\end{document}

